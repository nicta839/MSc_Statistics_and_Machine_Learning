% Options for packages loaded elsewhere
\PassOptionsToPackage{unicode}{hyperref}
\PassOptionsToPackage{hyphens}{url}
%
\documentclass[
]{article}
\usepackage{lmodern}
\usepackage{amssymb,amsmath}
\usepackage{ifxetex,ifluatex}
\ifnum 0\ifxetex 1\fi\ifluatex 1\fi=0 % if pdftex
  \usepackage[T1]{fontenc}
  \usepackage[utf8]{inputenc}
  \usepackage{textcomp} % provide euro and other symbols
\else % if luatex or xetex
  \usepackage{unicode-math}
  \defaultfontfeatures{Scale=MatchLowercase}
  \defaultfontfeatures[\rmfamily]{Ligatures=TeX,Scale=1}
\fi
% Use upquote if available, for straight quotes in verbatim environments
\IfFileExists{upquote.sty}{\usepackage{upquote}}{}
\IfFileExists{microtype.sty}{% use microtype if available
  \usepackage[]{microtype}
  \UseMicrotypeSet[protrusion]{basicmath} % disable protrusion for tt fonts
}{}
\makeatletter
\@ifundefined{KOMAClassName}{% if non-KOMA class
  \IfFileExists{parskip.sty}{%
    \usepackage{parskip}
  }{% else
    \setlength{\parindent}{0pt}
    \setlength{\parskip}{6pt plus 2pt minus 1pt}}
}{% if KOMA class
  \KOMAoptions{parskip=half}}
\makeatother
\usepackage{xcolor}
\IfFileExists{xurl.sty}{\usepackage{xurl}}{} % add URL line breaks if available
\IfFileExists{bookmark.sty}{\usepackage{bookmark}}{\usepackage{hyperref}}
\hypersetup{
  pdftitle={Lab 6 report},
  pdfauthor={Nicolas Taba \& Yuki Washio},
  hidelinks,
  pdfcreator={LaTeX via pandoc}}
\urlstyle{same} % disable monospaced font for URLs
\usepackage[margin=1in]{geometry}
\usepackage{color}
\usepackage{fancyvrb}
\newcommand{\VerbBar}{|}
\newcommand{\VERB}{\Verb[commandchars=\\\{\}]}
\DefineVerbatimEnvironment{Highlighting}{Verbatim}{commandchars=\\\{\}}
% Add ',fontsize=\small' for more characters per line
\usepackage{framed}
\definecolor{shadecolor}{RGB}{248,248,248}
\newenvironment{Shaded}{\begin{snugshade}}{\end{snugshade}}
\newcommand{\AlertTok}[1]{\textcolor[rgb]{0.94,0.16,0.16}{#1}}
\newcommand{\AnnotationTok}[1]{\textcolor[rgb]{0.56,0.35,0.01}{\textbf{\textit{#1}}}}
\newcommand{\AttributeTok}[1]{\textcolor[rgb]{0.77,0.63,0.00}{#1}}
\newcommand{\BaseNTok}[1]{\textcolor[rgb]{0.00,0.00,0.81}{#1}}
\newcommand{\BuiltInTok}[1]{#1}
\newcommand{\CharTok}[1]{\textcolor[rgb]{0.31,0.60,0.02}{#1}}
\newcommand{\CommentTok}[1]{\textcolor[rgb]{0.56,0.35,0.01}{\textit{#1}}}
\newcommand{\CommentVarTok}[1]{\textcolor[rgb]{0.56,0.35,0.01}{\textbf{\textit{#1}}}}
\newcommand{\ConstantTok}[1]{\textcolor[rgb]{0.00,0.00,0.00}{#1}}
\newcommand{\ControlFlowTok}[1]{\textcolor[rgb]{0.13,0.29,0.53}{\textbf{#1}}}
\newcommand{\DataTypeTok}[1]{\textcolor[rgb]{0.13,0.29,0.53}{#1}}
\newcommand{\DecValTok}[1]{\textcolor[rgb]{0.00,0.00,0.81}{#1}}
\newcommand{\DocumentationTok}[1]{\textcolor[rgb]{0.56,0.35,0.01}{\textbf{\textit{#1}}}}
\newcommand{\ErrorTok}[1]{\textcolor[rgb]{0.64,0.00,0.00}{\textbf{#1}}}
\newcommand{\ExtensionTok}[1]{#1}
\newcommand{\FloatTok}[1]{\textcolor[rgb]{0.00,0.00,0.81}{#1}}
\newcommand{\FunctionTok}[1]{\textcolor[rgb]{0.00,0.00,0.00}{#1}}
\newcommand{\ImportTok}[1]{#1}
\newcommand{\InformationTok}[1]{\textcolor[rgb]{0.56,0.35,0.01}{\textbf{\textit{#1}}}}
\newcommand{\KeywordTok}[1]{\textcolor[rgb]{0.13,0.29,0.53}{\textbf{#1}}}
\newcommand{\NormalTok}[1]{#1}
\newcommand{\OperatorTok}[1]{\textcolor[rgb]{0.81,0.36,0.00}{\textbf{#1}}}
\newcommand{\OtherTok}[1]{\textcolor[rgb]{0.56,0.35,0.01}{#1}}
\newcommand{\PreprocessorTok}[1]{\textcolor[rgb]{0.56,0.35,0.01}{\textit{#1}}}
\newcommand{\RegionMarkerTok}[1]{#1}
\newcommand{\SpecialCharTok}[1]{\textcolor[rgb]{0.00,0.00,0.00}{#1}}
\newcommand{\SpecialStringTok}[1]{\textcolor[rgb]{0.31,0.60,0.02}{#1}}
\newcommand{\StringTok}[1]{\textcolor[rgb]{0.31,0.60,0.02}{#1}}
\newcommand{\VariableTok}[1]{\textcolor[rgb]{0.00,0.00,0.00}{#1}}
\newcommand{\VerbatimStringTok}[1]{\textcolor[rgb]{0.31,0.60,0.02}{#1}}
\newcommand{\WarningTok}[1]{\textcolor[rgb]{0.56,0.35,0.01}{\textbf{\textit{#1}}}}
\usepackage{graphicx,grffile}
\makeatletter
\def\maxwidth{\ifdim\Gin@nat@width>\linewidth\linewidth\else\Gin@nat@width\fi}
\def\maxheight{\ifdim\Gin@nat@height>\textheight\textheight\else\Gin@nat@height\fi}
\makeatother
% Scale images if necessary, so that they will not overflow the page
% margins by default, and it is still possible to overwrite the defaults
% using explicit options in \includegraphics[width, height, ...]{}
\setkeys{Gin}{width=\maxwidth,height=\maxheight,keepaspectratio}
% Set default figure placement to htbp
\makeatletter
\def\fps@figure{htbp}
\makeatother
\setlength{\emergencystretch}{3em} % prevent overfull lines
\providecommand{\tightlist}{%
  \setlength{\itemsep}{0pt}\setlength{\parskip}{0pt}}
\setcounter{secnumdepth}{-\maxdimen} % remove section numbering

\title{Lab 6 report}
\author{Nicolas Taba \& Yuki Washio}
\date{14/12/2020}

\begin{document}
\maketitle

\hypertarget{question-1}{%
\subsection{Question 1}\label{question-1}}

\hypertarget{define-function-to-maximize}{%
\subsubsection{1 Define function to
maximize}\label{define-function-to-maximize}}

\begin{Shaded}
\begin{Highlighting}[]
\NormalTok{f_x <-}\StringTok{ }\ControlFlowTok{function}\NormalTok{(x) \{}
  \KeywordTok{return}\NormalTok{((x}\OperatorTok{^}\DecValTok{2} \OperatorTok{/}\StringTok{ }\KeywordTok{exp}\NormalTok{(x)) }\OperatorTok{-}\StringTok{ }\DecValTok{2} \OperatorTok{*}\StringTok{ }\KeywordTok{exp}\NormalTok{((}\OperatorTok{-}\NormalTok{(}\DecValTok{9} \OperatorTok{*}\StringTok{ }\KeywordTok{sin}\NormalTok{(x)) }\OperatorTok{/}\StringTok{ }\NormalTok{(x}\OperatorTok{^}\DecValTok{2} \OperatorTok{+}\StringTok{ }\NormalTok{x }\OperatorTok{+}\StringTok{ }\DecValTok{1}\NormalTok{))))}
\NormalTok{\}}
\end{Highlighting}
\end{Shaded}

\hypertarget{define-crossover}{%
\subsubsection{2 Define crossover()}\label{define-crossover}}

\begin{Shaded}
\begin{Highlighting}[]
\NormalTok{crossover <-}\StringTok{ }\ControlFlowTok{function}\NormalTok{(x, y) \{}
  \KeywordTok{return}\NormalTok{((x }\OperatorTok{+}\StringTok{ }\NormalTok{y) }\OperatorTok{/}\StringTok{ }\DecValTok{2}\NormalTok{)}
\NormalTok{\}}
\end{Highlighting}
\end{Shaded}

\hypertarget{define-mutate}{%
\subsubsection{3 Define mutate()}\label{define-mutate}}

\begin{Shaded}
\begin{Highlighting}[]
\NormalTok{mutate <-}\StringTok{ }\ControlFlowTok{function}\NormalTok{(x) \{}
  \KeywordTok{return}\NormalTok{((x}\OperatorTok{^}\DecValTok{2} \OperatorTok\StringTok{ }\DecValTok{30}\NormalTok{))}
\NormalTok{\}}
\end{Highlighting}
\end{Shaded}

\hypertarget{define-genetic-algorithm}{%
\subsubsection{4 Define genetic
algorithm}\label{define-genetic-algorithm}}

\begin{Shaded}
\begin{Highlighting}[]
\KeywordTok{library}\NormalTok{(ggplot2)}

\NormalTok{genetic <-}\StringTok{ }\ControlFlowTok{function}\NormalTok{(maxiter, mutprob) \{}
  \CommentTok{# a) plot function f in range [0,30]}
\NormalTok{  my_plot =}\StringTok{ }\KeywordTok{ggplot}\NormalTok{() }\OperatorTok{+}\StringTok{ }\KeywordTok{stat_function}\NormalTok{(}\DataTypeTok{fun=}\NormalTok{f_x) }\OperatorTok{+}\StringTok{ }\KeywordTok{xlim}\NormalTok{(}\DecValTok{0}\NormalTok{,}\DecValTok{30}\NormalTok{) }\OperatorTok{+}\StringTok{ }\KeywordTok{xlab}\NormalTok{(}\StringTok{"x"}\NormalTok{) }\OperatorTok{+}\StringTok{ }\KeywordTok{ylab}\NormalTok{(}\StringTok{"f(x)"}\NormalTok{)}
\NormalTok{  my_plot}
  \CommentTok{# b) initial population for the genetic algorithm as X = (0, 5, 10, 15, . . . , 30)}
\NormalTok{  initial_pop =}\StringTok{ }\KeywordTok{seq}\NormalTok{(}\DecValTok{0}\NormalTok{,}\DecValTok{30}\NormalTok{,}\DecValTok{5}\NormalTok{)}
\NormalTok{  population =}\StringTok{ }\NormalTok{initial_pop}
  \CommentTok{# c) compute vector Values that contain the function values for each in population}
\NormalTok{  function_values_pop =}\StringTok{ }\KeywordTok{f_x}\NormalTok{(population)}
  \CommentTok{# d) iterate}
  \ControlFlowTok{for}\NormalTok{ (i }\ControlFlowTok{in} \DecValTok{1}\OperatorTok{:}\NormalTok{maxiter) \{}
    \CommentTok{# sample two indexes from population}
\NormalTok{    parents =}\StringTok{ }\KeywordTok{sample}\NormalTok{(}\DecValTok{1}\OperatorTok{:}\KeywordTok{length}\NormalTok{(population), }\DecValTok{2}\NormalTok{)}
    \CommentTok{# select index with smallest objective function as victim}
\NormalTok{    victim =}\StringTok{ }\KeywordTok{order}\NormalTok{(function_values_pop)[}\DecValTok{1}\NormalTok{]}
    \CommentTok{# Parents are used to produce a new kid by crossover. Mutate this kid with probability mutprob (use crossover(), mutate()).}
\NormalTok{    new_kid =}\StringTok{ }\KeywordTok{crossover}\NormalTok{(parents[}\DecValTok{1}\NormalTok{], parents[}\DecValTok{2}\NormalTok{])}
    \ControlFlowTok{if}\NormalTok{ (}\KeywordTok{as.logical}\NormalTok{(}\KeywordTok{rbinom}\NormalTok{(}\DecValTok{1}\NormalTok{, }\DataTypeTok{size=}\DecValTok{1}\NormalTok{, }\DataTypeTok{prob=}\NormalTok{mutprob))) \{}
\NormalTok{      new_kid =}\StringTok{ }\KeywordTok{mutate}\NormalTok{(new_kid) }
\NormalTok{    \}}
    \CommentTok{# The victim is replaced by the kid in the population and the vector Values is updated.}
\NormalTok{    population[victim] =}\StringTok{ }\NormalTok{new_kid}
\NormalTok{    function_values_pop =}\StringTok{ }\KeywordTok{f_x}\NormalTok{(population)}
    \CommentTok{# The current maximal value of the objective function is saved.}
\NormalTok{    max_val =}\StringTok{ }\KeywordTok{max}\NormalTok{(function_values_pop)}
\NormalTok{  \}}
  \CommentTok{# e) Add the final observations to the current plot in another colour.}
\NormalTok{  final_plot =}\StringTok{ }\NormalTok{my_plot }\OperatorTok{+}\StringTok{ }\KeywordTok{geom_point}\NormalTok{(}\KeywordTok{aes}\NormalTok{(}\DataTypeTok{x=}\NormalTok{population, }\DataTypeTok{y=}\NormalTok{function_values_pop), }\DataTypeTok{color=}\StringTok{"red"}\NormalTok{)}
\NormalTok{  final_plot}
  \KeywordTok{return}\NormalTok{(}\KeywordTok{list}\NormalTok{(}\DataTypeTok{max_val=}\NormalTok{max_val, }\DataTypeTok{initial_pop=}\NormalTok{initial_pop, }\DataTypeTok{final_pop=}\NormalTok{population, }\DataTypeTok{final_fun_values=}\NormalTok{function_values_pop, }\DataTypeTok{plot=}\NormalTok{final_plot, }\DataTypeTok{maxiter=}\NormalTok{maxiter, }\DataTypeTok{mutprob=}\NormalTok{mutprob))}
\NormalTok{\}}
\end{Highlighting}
\end{Shaded}

We can observe a maximum value at around \(f(1.5) = 0.2\).

\hypertarget{try-out-different-parameter-combinations}{%
\subsubsection{5 Try out different parameter
combinations}\label{try-out-different-parameter-combinations}}

\includegraphics{report_lab6_files/figure-latex/unnamed-chunk-5-1.pdf}

\begin{verbatim}
## 1 
## maxiter:  10 
## mutprob:  0.1 
## initial_pop:  0 5 10 15 20 25 30 
## final_popp:  5.5 20.25 6.5 15 20 20.25 2.5 
## max_val:  -0.6380611 
## 
## 2 
## maxiter:  10 
## mutprob:  0.5 
## initial_pop:  0 5 10 15 20 25 30 
## final_popp:  6 9 20.25 15 3 2.5 2.5 
## max_val:  -0.6380611 
## 
## 3 
## maxiter:  10 
## mutprob:  0.9 
## initial_pop:  0 5 10 15 20 25 30 
## final_popp:  0.25 20.25 20.25 15 20 0.25 6.25 
## max_val:  -0.3179788 
## 
## 4 
## maxiter:  100 
## mutprob:  0.1 
## initial_pop:  0 5 10 15 20 25 30 
## final_popp:  1.5 1.5 1.5 4 2 2 1.5 
## max_val:  0.1998964 
## 
## 5 
## maxiter:  100 
## mutprob:  0.5 
## initial_pop:  0 5 10 15 20 25 30 
## final_popp:  1.5 1.5 1.5 1.5 0.25 2 2 
## max_val:  0.1998964 
## 
## 6 
## maxiter:  100 
## mutprob:  0.9 
## initial_pop:  0 5 10 15 20 25 30 
## final_popp:  12.25 0.25 0.25 0.25 0.25 2 0.25 
## max_val:  -0.07995372
\end{verbatim}

We are plotting all possible combinations. The first plot row shows all
plots where the maximum number of iterations is 10. The second plot row
shows all plots where the maximum number of iterations is 100. We can
observe that our genetic algorithm is not able to find the maximum with
just 10 iterations no matter how big the mutation probability is.
Though, we can also observe that given 100 iterations our genetic
algorithm returns good results in most cases. Setting mutation
probability to \(0.1\) or \(0.5\) returns the correct maximum value. For
mutation probability \(0.9\) we were not able to hit the true maximum.
This could be because of the high probability of not mutating the
crossover result.

\newpage

\hypertarget{question-2}{%
\subsection{Question 2}\label{question-2}}

\hypertarget{make-a-time-series-plot-of-the-data-with-respect-to-x}{%
\subsubsection{1: Make a time series plot of the data with respect to
X}\label{make-a-time-series-plot-of-the-data-with-respect-to-x}}

\begin{Shaded}
\begin{Highlighting}[]
\NormalTok{data <-}\StringTok{ }\KeywordTok{read.csv}\NormalTok{(}\StringTok{"physical1.csv"}\NormalTok{)}
\KeywordTok{library}\NormalTok{(ggplot2)}

\NormalTok{plot1 <-}\StringTok{ }\KeywordTok{ggplot}\NormalTok{(data)}\OperatorTok{+}
\StringTok{  }\KeywordTok{geom_line}\NormalTok{(}\KeywordTok{aes}\NormalTok{(}\DataTypeTok{x=}\NormalTok{X, }\DataTypeTok{y=}\NormalTok{Y, }\DataTypeTok{color =} \StringTok{"Y"}\NormalTok{))}\OperatorTok{+}
\StringTok{  }\KeywordTok{geom_line}\NormalTok{(}\KeywordTok{aes}\NormalTok{(}\DataTypeTok{x=}\NormalTok{X, }\DataTypeTok{y=}\NormalTok{Z, }\DataTypeTok{color =} \StringTok{"Z"}\NormalTok{))}\OperatorTok{+}
\StringTok{  }\KeywordTok{labs}\NormalTok{(}\DataTypeTok{x=}\StringTok{"X"}\NormalTok{, }\DataTypeTok{y=}\StringTok{""}\NormalTok{, }\DataTypeTok{title =} \StringTok{"Dependence of Z and Y versus X"}\NormalTok{)}

\NormalTok{plot2 <-}\StringTok{ }\KeywordTok{ggplot}\NormalTok{(data, }\KeywordTok{aes}\NormalTok{(}\DataTypeTok{x=}\NormalTok{Y, }\DataTypeTok{y=}\NormalTok{Z))}\OperatorTok{+}
\StringTok{  }\KeywordTok{geom_point}\NormalTok{()}\OperatorTok{+}
\StringTok{  }\KeywordTok{labs}\NormalTok{(}\DataTypeTok{title=}\StringTok{"Z versus Y"}\NormalTok{)}

\NormalTok{plot1}
\end{Highlighting}
\end{Shaded}

\includegraphics{report_lab6_files/figure-latex/unnamed-chunk-6-1.pdf}

\begin{Shaded}
\begin{Highlighting}[]
\NormalTok{plot2}
\end{Highlighting}
\end{Shaded}

\begin{verbatim}
## Warning: Removed 8 rows containing missing values (geom_point).
\end{verbatim}

\includegraphics{report_lab6_files/figure-latex/unnamed-chunk-6-2.pdf}

The two processes seem related when plotting the Y and Z variable
against X. Both processes have a decaying trend with sharp spikes that
seem to have a delay between them. The second plot represents a plot of
the Z variable bersus Y. There seems to be no clear correlation between
these two variables. We can assume that the two processes are
independent even if they stem from the X variable.

\hypertarget{derive-an-em-algorithm-that-estimates-lambda}{%
\subsubsection{2: derive an EM algorithm that estimates
lambda}\label{derive-an-em-algorithm-that-estimates-lambda}}

Variables Y and Z obey the following model:

\[Y_i \sim Exp\left(\frac{X_i}{\lambda}\right) \text{ which has a density function } f(Y_i) = \frac{X_i}{\lambda} \cdot \exp\left( -\frac{X_i}{\lambda} Y_i \right)\]
and

\[Z_i \sim Exp\left(\frac{X_i}{2\lambda}\right) \text{ which has a density function } f(Z_i) = \frac{X_i}{2\lambda} \cdot \exp\left( -\frac{X_i}{2\lambda} Z_i \right)\]

We then compute the likelihood of this model. The two variables are
assumed to be independent. This yields a likelihood that is the product
of the joint probabilities. From it, we then compute the following
log-likelihood:

\[\ln(L(\lambda | Y,Z))= \sum_{i=1}^{n} \left[\ln\left( \frac{X_i}{\lambda} \right) -\frac{X_i}{\lambda} Y_i \right] + \sum_{i=1}^{n} \left[ \ln\left( \frac{X_i}{2\lambda} \right) - \frac{X_i}{2\lambda} Z_i \right]\]

where the logarithm of a product becomes a sum over all points, the
exponential term is reduced to itself. We can further simplify this
expression by computing the first logarithm terms that do not depend on
\(Y_i\) and \(Z_i\) repsectively in each sum:

\[\ln(L(\lambda | Y,Z)) = \left[  - n\ln(\lambda) + \sum_{i=1}^{n}\ln(X_i) - \sum_{i=1}^{n} \frac{X_i}{\lambda} Y_i \right] + \left[  - n\ln(2\lambda) +  \sum_{i=1}^{n}\ln(X_i) - \sum_{i=1}^{n} \frac{X_i}{2\lambda} Z_i \right]\]

We have missing data in our \(Z_i\) set which makes the previous
equation impossible to compute. We must impute some data to replace the
missing data. A reasonable approach would be to replace missing values
for our random variable with the average value of \(Z_i\) at the missing
point \(X_i\). The mean value of \(Z_i\) at any \(X_i\) point, under the
exponential model, is : \(2\lambda/X_i\). The \(\lambda\) in this
expression is an estimate of it. In practice, we will choose the value
of lambda of the previous point. We thus separate the set into ``missing
set'' and ``non-missing set''. We can then compute the expected
log-likelihood from this information which adds another term to the
previous equation of the log-likelihood by separating the second term
into 2. We will be performing the second term sum of the log-likelihood
from 1 to \(m\) and then calculating the missing values from \(m+1\) to
\(n\) and replacing the sum over \(Z_i\) by our estimate:

\[ \begin{aligned}
E\Big[\ln(L(\lambda | Y,Z))\Big] =& \left[ \sum_{i=1}^{n}- n\ln(\lambda) + \ln(X_i)  + \sum_{i=1}^{n} \frac{X_i}{\lambda} Y_i \right] +
\left[ \sum_{j=1}^{m} - m\ln(2\lambda) + \ln(X_j) - \sum_{j=1}^{m} \frac{X_j}{2\lambda} Z_j \right] + \\
&\left[ \sum_{k=m+1}^{n} - (n-m)\ln(2\lambda) + \ln(X_k)  - \sum_{k=m+1}^{n} \frac{X_k}{2\lambda} \cdot \frac{2\lambda_k}{X_k} \right]
\end{aligned}\]

The \(X_k\) terms cancel out. We can now perform the maximization step
by taking the derivative of the expected log-likelihood with respect to
\(\lambda\) and setting it to zero. The middle term of each term of this
expression will be reduced to zero as they do not depend on \(\lambda\)
and we can factorize out \(1/\lambda\) from the subsequent equation:

\[\frac{d E\Big[\ln(L(\lambda | Y,Z))\Big]}{d\lambda} = 0\]

\[\Longrightarrow \left( - n + \sum_{i=1}^n \frac{X_i}{\lambda} Y_i - m + \sum_{j=1}^{m} \frac{X_j}{2\lambda} Z_j - (n-m) + (n-m) \frac{\lambda_k}{\lambda} \right) = 0 \]
We factor out again \(1/ \lambda\) and can compute the constants to get
\(-2n\). By rearranging the terms we obtain the maximum likelihood
estimate for \(\lambda\) as:

\[\hat\lambda_{ML} = \frac{ \sum_{i=1}^n X_i Y_i + \frac{1}{2} \sum_{j=1}^{m} X_j Z_j + (n-m) \lambda_k}{2n}\]

Where \(Z_j\) is our observed data, m is the number of observed points
for our random variable and \(\lambda_k\) will be the previous value.

\hypertarget{implement-the-previously-calculated-algorithm-in-r}{%
\subsubsection{3: Implement the previously calculated algorithm in
R}\label{implement-the-previously-calculated-algorithm-in-r}}

We implement the EM algorithm with the followinf restrictions:
\(\lambda_0 = 100\) and the stopping condition is when the change in
\(\lambda\) is smaller than \(10^{-3}\)

\begin{Shaded}
\begin{Highlighting}[]
\NormalTok{floglik <-}\StringTok{ }\ControlFlowTok{function}\NormalTok{(data, lambda, lambda_prev)\{}
\NormalTok{  X <-}\StringTok{ }\NormalTok{data[,}\DecValTok{1}\NormalTok{]}
\NormalTok{  Y <-}\StringTok{ }\NormalTok{data[,}\DecValTok{2}\NormalTok{]}
\NormalTok{  Z <-}\StringTok{ }\NormalTok{data[,}\DecValTok{3}\NormalTok{]}
  \CommentTok{# missing and not missing data}
\NormalTok{  index <-}\StringTok{ }\KeywordTok{is.na}\NormalTok{(Z)}
\NormalTok{  X_missing <-}\StringTok{ }\NormalTok{X[index]}
\NormalTok{  Z_not <-}\StringTok{ }\NormalTok{Z[}\OperatorTok{!}\NormalTok{index]}
\NormalTok{  X_not <-}\StringTok{ }\NormalTok{X[}\OperatorTok{!}\NormalTok{index]}
  \CommentTok{# n and m}
\NormalTok{  n <-}\StringTok{ }\KeywordTok{length}\NormalTok{(X)}
\NormalTok{  m <-}\StringTok{ }\KeywordTok{length}\NormalTok{(Z_not)}
  \CommentTok{# log likelihood}
\NormalTok{  Y_term <-}\StringTok{ }\NormalTok{(}\OperatorTok{-}\StringTok{ }\NormalTok{n }\OperatorTok{*}\StringTok{ }\KeywordTok{log}\NormalTok{(lambda)) }\OperatorTok{+}\StringTok{ }\KeywordTok{sum}\NormalTok{(}\KeywordTok{log}\NormalTok{(X)) }\OperatorTok{-}\StringTok{ }\KeywordTok{sum}\NormalTok{(X}\OperatorTok{*}\NormalTok{Y}\OperatorTok{/}\NormalTok{lambda)}
\NormalTok{  Z_not_term <-}\StringTok{ }\NormalTok{(}\OperatorTok{-}\StringTok{ }\NormalTok{m }\OperatorTok{*}\StringTok{ }\KeywordTok{log}\NormalTok{(}\DecValTok{2}\OperatorTok{*}\NormalTok{lambda)) }\OperatorTok{+}\StringTok{ }\KeywordTok{sum}\NormalTok{(}\KeywordTok{log}\NormalTok{(X_not)) }\OperatorTok{-}\StringTok{ }\KeywordTok{sum}\NormalTok{(X_not}\OperatorTok{*}\NormalTok{Z_not}\OperatorTok{/}\NormalTok{(}\DecValTok{2}\OperatorTok{*}\NormalTok{lambda))}
\NormalTok{  Z_missing_term <-}\StringTok{ }\NormalTok{(}\OperatorTok{-}\NormalTok{(n}\OperatorTok{-}\NormalTok{m)}\OperatorTok{*}\KeywordTok{log}\NormalTok{(}\DecValTok{2}\OperatorTok{*}\NormalTok{lambda)) }\OperatorTok{+}\StringTok{ }\KeywordTok{sum}\NormalTok{(}\KeywordTok{log}\NormalTok{(X_missing)) }\OperatorTok{-}\StringTok{ }\NormalTok{((n}\OperatorTok{-}\NormalTok{m)}\OperatorTok{*}\NormalTok{lambda_prev}\OperatorTok{/}\NormalTok{lambda)}
  
\NormalTok{  log_like <-}\StringTok{ }\NormalTok{Y_term }\OperatorTok{+}\StringTok{ }\NormalTok{Z_not_term }\OperatorTok{+}\StringTok{ }\NormalTok{Z_missing_term}
  \KeywordTok{return}\NormalTok{(log_like)}
\NormalTok{\}}

\NormalTok{EM_algo <-}\StringTok{ }\ControlFlowTok{function}\NormalTok{(data, eps, k_max, lambda_}\DecValTok{0}\NormalTok{)\{}
\NormalTok{  X <-}\StringTok{ }\NormalTok{data[,}\DecValTok{1}\NormalTok{]}
\NormalTok{  Y <-}\StringTok{ }\NormalTok{data[,}\DecValTok{2}\NormalTok{]}
\NormalTok{  Z <-}\StringTok{ }\NormalTok{data[,}\DecValTok{3}\NormalTok{]}
  \CommentTok{# missing and not missing data}
\NormalTok{  index <-}\StringTok{ }\KeywordTok{is.na}\NormalTok{(Z)}
\NormalTok{  X_missing <-}\StringTok{ }\NormalTok{X[index]}
\NormalTok{  Z_not <-}\StringTok{ }\NormalTok{Z[}\OperatorTok{!}\NormalTok{index]}
\NormalTok{  X_not <-}\StringTok{ }\NormalTok{X[}\OperatorTok{!}\NormalTok{index]}
  \CommentTok{# n and m}
\NormalTok{  n <-}\StringTok{ }\KeywordTok{length}\NormalTok{(X)}
\NormalTok{  m <-}\StringTok{ }\KeywordTok{length}\NormalTok{(Z_not)}
  
  \CommentTok{#initialize everything}
\NormalTok{  k <-}\StringTok{ }\DecValTok{1}
\NormalTok{  lambda_current <-}\StringTok{ }\NormalTok{lambda_}\DecValTok{0}
\NormalTok{  lambda_prev <-}\StringTok{ }\NormalTok{lambda_}\DecValTok{0}\OperatorTok{+}\DecValTok{1}\OperatorTok{+}\DecValTok{100}\OperatorTok{*}\NormalTok{eps}
\NormalTok{  llvalcurr <-}\StringTok{ }\KeywordTok{floglik}\NormalTok{(data, lambda_current, lambda_prev)}
  
  \CommentTok{# EM algorithm}
  \ControlFlowTok{while}\NormalTok{( (}\KeywordTok{abs}\NormalTok{(lambda_prev}\OperatorTok{-}\NormalTok{lambda_current) }\OperatorTok{>}\StringTok{ }\NormalTok{eps) }\OperatorTok{&&}\StringTok{  }\NormalTok{(k }\OperatorTok{<}\StringTok{ }\NormalTok{(k_max}\OperatorTok{+}\DecValTok{1}\NormalTok{)) )\{}
\NormalTok{    lambda_prev <-}\StringTok{ }\NormalTok{lambda_current}
    
    \CommentTok{# EM step}
\NormalTok{    lambda_current <-}\StringTok{ }\NormalTok{(}\KeywordTok{sum}\NormalTok{(X}\OperatorTok{*}\NormalTok{Y) }\OperatorTok{+}\StringTok{ }\KeywordTok{sum}\NormalTok{(X_not}\OperatorTok{*}\NormalTok{Z_not}\OperatorTok{/}\DecValTok{2}\NormalTok{) }\OperatorTok{+}\StringTok{ }\NormalTok{(n}\OperatorTok{-}\NormalTok{m)}\OperatorTok{*}\NormalTok{lambda_current)}\OperatorTok{/}\NormalTok{(}\DecValTok{2}\OperatorTok{*}\NormalTok{n)}
    \CommentTok{# new log-likelihood}
\NormalTok{    llvalcurr <-}\StringTok{ }\KeywordTok{floglik}\NormalTok{(data, lambda_current, lambda_prev)}
\NormalTok{    k <-}\StringTok{ }\NormalTok{k}\OperatorTok{+}\DecValTok{1}
\NormalTok{  \}}
\NormalTok{  results <-}\StringTok{ }\KeywordTok{list}\NormalTok{(}\DataTypeTok{steps =}\NormalTok{ k, }\DataTypeTok{lambda =}\NormalTok{ lambda_current)}
  \KeywordTok{return}\NormalTok{(results)}
\NormalTok{\}}

\NormalTok{epsilon <-}\StringTok{ }\FloatTok{0.001}
\NormalTok{my_lambda_}\DecValTok{0}\NormalTok{ <-}\StringTok{ }\DecValTok{100}
\NormalTok{my_k_max <-}\StringTok{ }\DecValTok{100000}

\NormalTok{res <-}\StringTok{ }\KeywordTok{EM_algo}\NormalTok{(data, }\DataTypeTok{eps =}\NormalTok{ epsilon, }\DataTypeTok{k_max =}\NormalTok{ my_k_max, }\DataTypeTok{lambda_0 =}\NormalTok{ my_lambda_}\DecValTok{0}\NormalTok{)}
\NormalTok{res}
\end{Highlighting}
\end{Shaded}

\begin{verbatim}
## $steps
## [1] 6
## 
## $lambda
## [1] 10.69566
\end{verbatim}

the optimal \(\lambda\) is 10.696 and it took 6 iterations for the
algorithm to satisfy our convergence criterion.

\hypertarget{plot-ey-and-ez}{%
\subsubsection{4 Plot E{[}Y{]} and E{[}Z{]}}\label{plot-ey-and-ez}}

\begin{Shaded}
\begin{Highlighting}[]
\CommentTok{# Expectation value of random exponential variable is the inverse of its parameter}
\NormalTok{E_Y <-}\StringTok{ }\NormalTok{res[[}\DecValTok{2}\NormalTok{]]}\OperatorTok{/}\NormalTok{data[,}\DecValTok{1}\NormalTok{]}
\NormalTok{E_Z <-}\StringTok{ }\NormalTok{(}\DecValTok{2}\OperatorTok{*}\NormalTok{res[[}\DecValTok{2}\NormalTok{]])}\OperatorTok{/}\NormalTok{data[,}\DecValTok{1}\NormalTok{]}
\NormalTok{df_plot <-}\StringTok{ }\KeywordTok{data.frame}\NormalTok{(}\DataTypeTok{X =}\NormalTok{ data[,}\DecValTok{1}\NormalTok{], }\DataTypeTok{Y =}\NormalTok{ data[,}\DecValTok{2}\NormalTok{], }\DataTypeTok{Z =}\NormalTok{ data[,}\DecValTok{3}\NormalTok{], }\DataTypeTok{E_Y =}\NormalTok{ E_Y, }\DataTypeTok{E_Z =}\NormalTok{ E_Z)}


\NormalTok{plot3 <-}\StringTok{ }\KeywordTok{ggplot}\NormalTok{(df_plot)}\OperatorTok{+}
\StringTok{  }\KeywordTok{geom_line}\NormalTok{(}\KeywordTok{aes}\NormalTok{(}\DataTypeTok{x=}\NormalTok{X, }\DataTypeTok{y=}\NormalTok{Y, }\DataTypeTok{color =} \StringTok{"Y"}\NormalTok{))}\OperatorTok{+}
\StringTok{  }\KeywordTok{geom_line}\NormalTok{(}\KeywordTok{aes}\NormalTok{(}\DataTypeTok{x=}\NormalTok{X, }\DataTypeTok{y=}\NormalTok{Z, }\DataTypeTok{color =} \StringTok{"Z"}\NormalTok{))}\OperatorTok{+}
\StringTok{  }\KeywordTok{geom_point}\NormalTok{(}\KeywordTok{aes}\NormalTok{(}\DataTypeTok{x=}\NormalTok{ X, }\DataTypeTok{y=}\NormalTok{E_Y, }\DataTypeTok{color =} \StringTok{"E[Y]"}\NormalTok{))}\OperatorTok{+}
\StringTok{  }\KeywordTok{geom_point}\NormalTok{(}\KeywordTok{aes}\NormalTok{(}\DataTypeTok{x=}\NormalTok{X, }\DataTypeTok{y=}\NormalTok{E_Z, }\DataTypeTok{color =} \StringTok{"E[Z]"}\NormalTok{))}\OperatorTok{+}
\StringTok{  }\KeywordTok{labs}\NormalTok{(}\DataTypeTok{x=}\StringTok{"X"}\NormalTok{, }\DataTypeTok{y=}\StringTok{""}\NormalTok{, }\DataTypeTok{title =} \StringTok{"Z and Y versus X and estimates"}\NormalTok{)}

\NormalTok{plot3}
\end{Highlighting}
\end{Shaded}

\includegraphics{report_lab6_files/figure-latex/unnamed-chunk-8-1.pdf}

The expected value of our variables seem to follow the general trend of
the noisy data.. The Y data has peaks that are smaller than our Z data.
The difference in expectation values between the two processes probably
takes those differences into account.

\newpage

\hypertarget{appendix-all-code-for-this-report}{%
\subsection{Appendix: All code for this
report}\label{appendix-all-code-for-this-report}}

\begin{Shaded}
\begin{Highlighting}[]
\NormalTok{knitr}\OperatorTok{::}\NormalTok{opts_chunk}\OperatorTok{$}\KeywordTok{set}\NormalTok{(}\DataTypeTok{echo =} \OtherTok{TRUE}\NormalTok{)}
\NormalTok{f_x <-}\StringTok{ }\ControlFlowTok{function}\NormalTok{(x) \{}
  \KeywordTok{return}\NormalTok{((x}\OperatorTok{^}\DecValTok{2} \OperatorTok{/}\StringTok{ }\KeywordTok{exp}\NormalTok{(x)) }\OperatorTok{-}\StringTok{ }\DecValTok{2} \OperatorTok{*}\StringTok{ }\KeywordTok{exp}\NormalTok{((}\OperatorTok{-}\NormalTok{(}\DecValTok{9} \OperatorTok{*}\StringTok{ }\KeywordTok{sin}\NormalTok{(x)) }\OperatorTok{/}\StringTok{ }\NormalTok{(x}\OperatorTok{^}\DecValTok{2} \OperatorTok{+}\StringTok{ }\NormalTok{x }\OperatorTok{+}\StringTok{ }\DecValTok{1}\NormalTok{))))}
\NormalTok{\}}
\NormalTok{crossover <-}\StringTok{ }\ControlFlowTok{function}\NormalTok{(x, y) \{}
  \KeywordTok{return}\NormalTok{((x }\OperatorTok{+}\StringTok{ }\NormalTok{y) }\OperatorTok{/}\StringTok{ }\DecValTok{2}\NormalTok{)}
\NormalTok{\}}
\NormalTok{mutate <-}\StringTok{ }\ControlFlowTok{function}\NormalTok{(x) \{}
  \KeywordTok{return}\NormalTok{((x}\OperatorTok{^}\DecValTok{2} \OperatorTok\StringTok{ }\DecValTok{30}\NormalTok{))}
\NormalTok{\}}
\KeywordTok{library}\NormalTok{(ggplot2)}

\NormalTok{genetic <-}\StringTok{ }\ControlFlowTok{function}\NormalTok{(maxiter, mutprob) \{}
  \CommentTok{# a) plot function f in range [0,30]}
\NormalTok{  my_plot =}\StringTok{ }\KeywordTok{ggplot}\NormalTok{() }\OperatorTok{+}\StringTok{ }\KeywordTok{stat_function}\NormalTok{(}\DataTypeTok{fun=}\NormalTok{f_x) }\OperatorTok{+}\StringTok{ }\KeywordTok{xlim}\NormalTok{(}\DecValTok{0}\NormalTok{,}\DecValTok{30}\NormalTok{) }\OperatorTok{+}\StringTok{ }\KeywordTok{xlab}\NormalTok{(}\StringTok{"x"}\NormalTok{) }\OperatorTok{+}\StringTok{ }\KeywordTok{ylab}\NormalTok{(}\StringTok{"f(x)"}\NormalTok{)}
\NormalTok{  my_plot}
  \CommentTok{# b) initial population for the genetic algorithm as X = (0, 5, 10, 15, . . . , 30)}
\NormalTok{  initial_pop =}\StringTok{ }\KeywordTok{seq}\NormalTok{(}\DecValTok{0}\NormalTok{,}\DecValTok{30}\NormalTok{,}\DecValTok{5}\NormalTok{)}
\NormalTok{  population =}\StringTok{ }\NormalTok{initial_pop}
  \CommentTok{# c) compute vector Values that contain the function values for each in population}
\NormalTok{  function_values_pop =}\StringTok{ }\KeywordTok{f_x}\NormalTok{(population)}
  \CommentTok{# d) iterate}
  \ControlFlowTok{for}\NormalTok{ (i }\ControlFlowTok{in} \DecValTok{1}\OperatorTok{:}\NormalTok{maxiter) \{}
    \CommentTok{# sample two indexes from population}
\NormalTok{    parents =}\StringTok{ }\KeywordTok{sample}\NormalTok{(}\DecValTok{1}\OperatorTok{:}\KeywordTok{length}\NormalTok{(population), }\DecValTok{2}\NormalTok{)}
    \CommentTok{# select index with smallest objective function as victim}
\NormalTok{    victim =}\StringTok{ }\KeywordTok{order}\NormalTok{(function_values_pop)[}\DecValTok{1}\NormalTok{]}
    \CommentTok{# Parents are used to produce a new kid by crossover. Mutate this kid with probability mutprob (use crossover(), mutate()).}
\NormalTok{    new_kid =}\StringTok{ }\KeywordTok{crossover}\NormalTok{(parents[}\DecValTok{1}\NormalTok{], parents[}\DecValTok{2}\NormalTok{])}
    \ControlFlowTok{if}\NormalTok{ (}\KeywordTok{as.logical}\NormalTok{(}\KeywordTok{rbinom}\NormalTok{(}\DecValTok{1}\NormalTok{, }\DataTypeTok{size=}\DecValTok{1}\NormalTok{, }\DataTypeTok{prob=}\NormalTok{mutprob))) \{}
\NormalTok{      new_kid =}\StringTok{ }\KeywordTok{mutate}\NormalTok{(new_kid) }
\NormalTok{    \}}
    \CommentTok{# The victim is replaced by the kid in the population and the vector Values is updated.}
\NormalTok{    population[victim] =}\StringTok{ }\NormalTok{new_kid}
\NormalTok{    function_values_pop =}\StringTok{ }\KeywordTok{f_x}\NormalTok{(population)}
    \CommentTok{# The current maximal value of the objective function is saved.}
\NormalTok{    max_val =}\StringTok{ }\KeywordTok{max}\NormalTok{(function_values_pop)}
\NormalTok{  \}}
  \CommentTok{# e) Add the final observations to the current plot in another colour.}
\NormalTok{  final_plot =}\StringTok{ }\NormalTok{my_plot }\OperatorTok{+}\StringTok{ }\KeywordTok{geom_point}\NormalTok{(}\KeywordTok{aes}\NormalTok{(}\DataTypeTok{x=}\NormalTok{population, }\DataTypeTok{y=}\NormalTok{function_values_pop), }\DataTypeTok{color=}\StringTok{"red"}\NormalTok{)}
\NormalTok{  final_plot}
  \KeywordTok{return}\NormalTok{(}\KeywordTok{list}\NormalTok{(}\DataTypeTok{max_val=}\NormalTok{max_val, }\DataTypeTok{initial_pop=}\NormalTok{initial_pop, }\DataTypeTok{final_pop=}\NormalTok{population, }\DataTypeTok{final_fun_values=}\NormalTok{function_values_pop, }\DataTypeTok{plot=}\NormalTok{final_plot, }\DataTypeTok{maxiter=}\NormalTok{maxiter, }\DataTypeTok{mutprob=}\NormalTok{mutprob))}
\NormalTok{\}}
\NormalTok{max_iterations =}\StringTok{ }\KeywordTok{c}\NormalTok{(}\DecValTok{10}\NormalTok{, }\DecValTok{100}\NormalTok{)}
\NormalTok{mutation_probabilities =}\StringTok{ }\KeywordTok{c}\NormalTok{(}\FloatTok{0.1}\NormalTok{, }\FloatTok{0.5}\NormalTok{, }\FloatTok{0.9}\NormalTok{)}

\KeywordTok{set.seed}\NormalTok{(}\DecValTok{1234567890}\NormalTok{)}
\NormalTok{res1 =}\StringTok{ }\KeywordTok{genetic}\NormalTok{(}\DecValTok{10}\NormalTok{, }\FloatTok{0.1}\NormalTok{)}
\NormalTok{res2 =}\StringTok{ }\KeywordTok{genetic}\NormalTok{(}\DecValTok{10}\NormalTok{, }\FloatTok{0.5}\NormalTok{)}
\NormalTok{res3 =}\StringTok{ }\KeywordTok{genetic}\NormalTok{(}\DecValTok{10}\NormalTok{, }\FloatTok{0.9}\NormalTok{)}
\NormalTok{res4 =}\StringTok{ }\KeywordTok{genetic}\NormalTok{(}\DecValTok{100}\NormalTok{, }\FloatTok{0.1}\NormalTok{)}
\NormalTok{res5 =}\StringTok{ }\KeywordTok{genetic}\NormalTok{(}\DecValTok{100}\NormalTok{, }\FloatTok{0.5}\NormalTok{)}
\NormalTok{res6 =}\StringTok{ }\KeywordTok{genetic}\NormalTok{(}\DecValTok{100}\NormalTok{, }\FloatTok{0.9}\NormalTok{)}
\NormalTok{results =}\StringTok{ }\KeywordTok{list}\NormalTok{(res1, res2, res3, res4, res5, res6)}

\KeywordTok{library}\NormalTok{(gridExtra)}
\KeywordTok{grid.arrange}\NormalTok{(res1}\OperatorTok{$}\NormalTok{plot, res2}\OperatorTok{$}\NormalTok{plot, res3}\OperatorTok{$}\NormalTok{plot, res4}\OperatorTok{$}\NormalTok{plot, res5}\OperatorTok{$}\NormalTok{plot, res6}\OperatorTok{$}\NormalTok{plot, }\DataTypeTok{ncol=}\DecValTok{3}\NormalTok{)}

\ControlFlowTok{for}\NormalTok{ (i }\ControlFlowTok{in} \DecValTok{1}\OperatorTok{:}\KeywordTok{length}\NormalTok{(results)) \{}
  \KeywordTok{cat}\NormalTok{(i, }\StringTok{"}\CharTok{\textbackslash{}n}\StringTok{maxiter: "}\NormalTok{, results[[i]]}\OperatorTok{$}\NormalTok{maxiter, }\StringTok{"}\CharTok{\textbackslash{}n}\StringTok{mutprob: "}\NormalTok{, results[[i]]}\OperatorTok{$}\NormalTok{mutprob, }\StringTok{"}\CharTok{\textbackslash{}n}\StringTok{initial_pop: "}\NormalTok{, results[[i]]}\OperatorTok{$}\NormalTok{initial_pop, }\StringTok{"}\CharTok{\textbackslash{}n}\StringTok{final_popp: "}\NormalTok{, results[[i]]}\OperatorTok{$}\NormalTok{final_pop, }\StringTok{"}\CharTok{\textbackslash{}n}\StringTok{max_val: "}\NormalTok{, results[[i]]}\OperatorTok{$}\NormalTok{max_val, }\StringTok{"}\CharTok{\textbackslash{}n\textbackslash{}n}\StringTok{"}\NormalTok{)}
\NormalTok{\}}
\NormalTok{data <-}\StringTok{ }\KeywordTok{read.csv}\NormalTok{(}\StringTok{"physical1.csv"}\NormalTok{)}
\KeywordTok{library}\NormalTok{(ggplot2)}

\NormalTok{plot1 <-}\StringTok{ }\KeywordTok{ggplot}\NormalTok{(data)}\OperatorTok{+}
\StringTok{  }\KeywordTok{geom_line}\NormalTok{(}\KeywordTok{aes}\NormalTok{(}\DataTypeTok{x=}\NormalTok{X, }\DataTypeTok{y=}\NormalTok{Y, }\DataTypeTok{color =} \StringTok{"Y"}\NormalTok{))}\OperatorTok{+}
\StringTok{  }\KeywordTok{geom_line}\NormalTok{(}\KeywordTok{aes}\NormalTok{(}\DataTypeTok{x=}\NormalTok{X, }\DataTypeTok{y=}\NormalTok{Z, }\DataTypeTok{color =} \StringTok{"Z"}\NormalTok{))}\OperatorTok{+}
\StringTok{  }\KeywordTok{labs}\NormalTok{(}\DataTypeTok{x=}\StringTok{"X"}\NormalTok{, }\DataTypeTok{y=}\StringTok{""}\NormalTok{, }\DataTypeTok{title =} \StringTok{"Dependence of Z and Y versus X"}\NormalTok{)}

\NormalTok{plot2 <-}\StringTok{ }\KeywordTok{ggplot}\NormalTok{(data, }\KeywordTok{aes}\NormalTok{(}\DataTypeTok{x=}\NormalTok{Y, }\DataTypeTok{y=}\NormalTok{Z))}\OperatorTok{+}
\StringTok{  }\KeywordTok{geom_point}\NormalTok{()}\OperatorTok{+}
\StringTok{  }\KeywordTok{labs}\NormalTok{(}\DataTypeTok{title=}\StringTok{"Z versus Y"}\NormalTok{)}

\NormalTok{plot1}
\NormalTok{plot2}

\NormalTok{floglik <-}\StringTok{ }\ControlFlowTok{function}\NormalTok{(data, lambda, lambda_prev)\{}
\NormalTok{  X <-}\StringTok{ }\NormalTok{data[,}\DecValTok{1}\NormalTok{]}
\NormalTok{  Y <-}\StringTok{ }\NormalTok{data[,}\DecValTok{2}\NormalTok{]}
\NormalTok{  Z <-}\StringTok{ }\NormalTok{data[,}\DecValTok{3}\NormalTok{]}
  \CommentTok{# missing and not missing data}
\NormalTok{  index <-}\StringTok{ }\KeywordTok{is.na}\NormalTok{(Z)}
\NormalTok{  X_missing <-}\StringTok{ }\NormalTok{X[index]}
\NormalTok{  Z_not <-}\StringTok{ }\NormalTok{Z[}\OperatorTok{!}\NormalTok{index]}
\NormalTok{  X_not <-}\StringTok{ }\NormalTok{X[}\OperatorTok{!}\NormalTok{index]}
  \CommentTok{# n and m}
\NormalTok{  n <-}\StringTok{ }\KeywordTok{length}\NormalTok{(X)}
\NormalTok{  m <-}\StringTok{ }\KeywordTok{length}\NormalTok{(Z_not)}
  \CommentTok{# log likelihood}
\NormalTok{  Y_term <-}\StringTok{ }\NormalTok{(}\OperatorTok{-}\StringTok{ }\NormalTok{n }\OperatorTok{*}\StringTok{ }\KeywordTok{log}\NormalTok{(lambda)) }\OperatorTok{+}\StringTok{ }\KeywordTok{sum}\NormalTok{(}\KeywordTok{log}\NormalTok{(X)) }\OperatorTok{-}\StringTok{ }\KeywordTok{sum}\NormalTok{(X}\OperatorTok{*}\NormalTok{Y}\OperatorTok{/}\NormalTok{lambda)}
\NormalTok{  Z_not_term <-}\StringTok{ }\NormalTok{(}\OperatorTok{-}\StringTok{ }\NormalTok{m }\OperatorTok{*}\StringTok{ }\KeywordTok{log}\NormalTok{(}\DecValTok{2}\OperatorTok{*}\NormalTok{lambda)) }\OperatorTok{+}\StringTok{ }\KeywordTok{sum}\NormalTok{(}\KeywordTok{log}\NormalTok{(X_not)) }\OperatorTok{-}\StringTok{ }\KeywordTok{sum}\NormalTok{(X_not}\OperatorTok{*}\NormalTok{Z_not}\OperatorTok{/}\NormalTok{(}\DecValTok{2}\OperatorTok{*}\NormalTok{lambda))}
\NormalTok{  Z_missing_term <-}\StringTok{ }\NormalTok{(}\OperatorTok{-}\NormalTok{(n}\OperatorTok{-}\NormalTok{m)}\OperatorTok{*}\KeywordTok{log}\NormalTok{(}\DecValTok{2}\OperatorTok{*}\NormalTok{lambda)) }\OperatorTok{+}\StringTok{ }\KeywordTok{sum}\NormalTok{(}\KeywordTok{log}\NormalTok{(X_missing)) }\OperatorTok{-}\StringTok{ }\NormalTok{((n}\OperatorTok{-}\NormalTok{m)}\OperatorTok{*}\NormalTok{lambda_prev}\OperatorTok{/}\NormalTok{lambda)}
  
\NormalTok{  log_like <-}\StringTok{ }\NormalTok{Y_term }\OperatorTok{+}\StringTok{ }\NormalTok{Z_not_term }\OperatorTok{+}\StringTok{ }\NormalTok{Z_missing_term}
  \KeywordTok{return}\NormalTok{(log_like)}
\NormalTok{\}}

\NormalTok{EM_algo <-}\StringTok{ }\ControlFlowTok{function}\NormalTok{(data, eps, k_max, lambda_}\DecValTok{0}\NormalTok{)\{}
\NormalTok{  X <-}\StringTok{ }\NormalTok{data[,}\DecValTok{1}\NormalTok{]}
\NormalTok{  Y <-}\StringTok{ }\NormalTok{data[,}\DecValTok{2}\NormalTok{]}
\NormalTok{  Z <-}\StringTok{ }\NormalTok{data[,}\DecValTok{3}\NormalTok{]}
  \CommentTok{# missing and not missing data}
\NormalTok{  index <-}\StringTok{ }\KeywordTok{is.na}\NormalTok{(Z)}
\NormalTok{  X_missing <-}\StringTok{ }\NormalTok{X[index]}
\NormalTok{  Z_not <-}\StringTok{ }\NormalTok{Z[}\OperatorTok{!}\NormalTok{index]}
\NormalTok{  X_not <-}\StringTok{ }\NormalTok{X[}\OperatorTok{!}\NormalTok{index]}
  \CommentTok{# n and m}
\NormalTok{  n <-}\StringTok{ }\KeywordTok{length}\NormalTok{(X)}
\NormalTok{  m <-}\StringTok{ }\KeywordTok{length}\NormalTok{(Z_not)}
  
  \CommentTok{#initialize everything}
\NormalTok{  k <-}\StringTok{ }\DecValTok{1}
\NormalTok{  lambda_current <-}\StringTok{ }\NormalTok{lambda_}\DecValTok{0}
\NormalTok{  lambda_prev <-}\StringTok{ }\NormalTok{lambda_}\DecValTok{0}\OperatorTok{+}\DecValTok{1}\OperatorTok{+}\DecValTok{100}\OperatorTok{*}\NormalTok{eps}
\NormalTok{  llvalcurr <-}\StringTok{ }\KeywordTok{floglik}\NormalTok{(data, lambda_current, lambda_prev)}
  
  \CommentTok{# EM algorithm}
  \ControlFlowTok{while}\NormalTok{( (}\KeywordTok{abs}\NormalTok{(lambda_prev}\OperatorTok{-}\NormalTok{lambda_current) }\OperatorTok{>}\StringTok{ }\NormalTok{eps) }\OperatorTok{&&}\StringTok{  }\NormalTok{(k }\OperatorTok{<}\StringTok{ }\NormalTok{(k_max}\OperatorTok{+}\DecValTok{1}\NormalTok{)) )\{}
\NormalTok{    lambda_prev <-}\StringTok{ }\NormalTok{lambda_current}
    
    \CommentTok{# EM step}
\NormalTok{    lambda_current <-}\StringTok{ }\NormalTok{(}\KeywordTok{sum}\NormalTok{(X}\OperatorTok{*}\NormalTok{Y) }\OperatorTok{+}\StringTok{ }\KeywordTok{sum}\NormalTok{(X_not}\OperatorTok{*}\NormalTok{Z_not}\OperatorTok{/}\DecValTok{2}\NormalTok{) }\OperatorTok{+}\StringTok{ }\NormalTok{(n}\OperatorTok{-}\NormalTok{m)}\OperatorTok{*}\NormalTok{lambda_current)}\OperatorTok{/}\NormalTok{(}\DecValTok{2}\OperatorTok{*}\NormalTok{n)}
    \CommentTok{# new log-likelihood}
\NormalTok{    llvalcurr <-}\StringTok{ }\KeywordTok{floglik}\NormalTok{(data, lambda_current, lambda_prev)}
\NormalTok{    k <-}\StringTok{ }\NormalTok{k}\OperatorTok{+}\DecValTok{1}
\NormalTok{  \}}
\NormalTok{  results <-}\StringTok{ }\KeywordTok{list}\NormalTok{(}\DataTypeTok{steps =}\NormalTok{ k, }\DataTypeTok{lambda =}\NormalTok{ lambda_current)}
  \KeywordTok{return}\NormalTok{(results)}
\NormalTok{\}}

\NormalTok{epsilon <-}\StringTok{ }\FloatTok{0.001}
\NormalTok{my_lambda_}\DecValTok{0}\NormalTok{ <-}\StringTok{ }\DecValTok{100}
\NormalTok{my_k_max <-}\StringTok{ }\DecValTok{100000}

\NormalTok{res <-}\StringTok{ }\KeywordTok{EM_algo}\NormalTok{(data, }\DataTypeTok{eps =}\NormalTok{ epsilon, }\DataTypeTok{k_max =}\NormalTok{ my_k_max, }\DataTypeTok{lambda_0 =}\NormalTok{ my_lambda_}\DecValTok{0}\NormalTok{)}
\NormalTok{res}
\CommentTok{# Expectation value of random exponential variable is the inverse of its parameter}
\NormalTok{E_Y <-}\StringTok{ }\NormalTok{res[[}\DecValTok{2}\NormalTok{]]}\OperatorTok{/}\NormalTok{data[,}\DecValTok{1}\NormalTok{]}
\NormalTok{E_Z <-}\StringTok{ }\NormalTok{(}\DecValTok{2}\OperatorTok{*}\NormalTok{res[[}\DecValTok{2}\NormalTok{]])}\OperatorTok{/}\NormalTok{data[,}\DecValTok{1}\NormalTok{]}
\NormalTok{df_plot <-}\StringTok{ }\KeywordTok{data.frame}\NormalTok{(}\DataTypeTok{X =}\NormalTok{ data[,}\DecValTok{1}\NormalTok{], }\DataTypeTok{Y =}\NormalTok{ data[,}\DecValTok{2}\NormalTok{], }\DataTypeTok{Z =}\NormalTok{ data[,}\DecValTok{3}\NormalTok{], }\DataTypeTok{E_Y =}\NormalTok{ E_Y, }\DataTypeTok{E_Z =}\NormalTok{ E_Z)}


\NormalTok{plot3 <-}\StringTok{ }\KeywordTok{ggplot}\NormalTok{(df_plot)}\OperatorTok{+}
\StringTok{  }\KeywordTok{geom_line}\NormalTok{(}\KeywordTok{aes}\NormalTok{(}\DataTypeTok{x=}\NormalTok{X, }\DataTypeTok{y=}\NormalTok{Y, }\DataTypeTok{color =} \StringTok{"Y"}\NormalTok{))}\OperatorTok{+}
\StringTok{  }\KeywordTok{geom_line}\NormalTok{(}\KeywordTok{aes}\NormalTok{(}\DataTypeTok{x=}\NormalTok{X, }\DataTypeTok{y=}\NormalTok{Z, }\DataTypeTok{color =} \StringTok{"Z"}\NormalTok{))}\OperatorTok{+}
\StringTok{  }\KeywordTok{geom_point}\NormalTok{(}\KeywordTok{aes}\NormalTok{(}\DataTypeTok{x=}\NormalTok{ X, }\DataTypeTok{y=}\NormalTok{E_Y, }\DataTypeTok{color =} \StringTok{"E[Y]"}\NormalTok{))}\OperatorTok{+}
\StringTok{  }\KeywordTok{geom_point}\NormalTok{(}\KeywordTok{aes}\NormalTok{(}\DataTypeTok{x=}\NormalTok{X, }\DataTypeTok{y=}\NormalTok{E_Z, }\DataTypeTok{color =} \StringTok{"E[Z]"}\NormalTok{))}\OperatorTok{+}
\StringTok{  }\KeywordTok{labs}\NormalTok{(}\DataTypeTok{x=}\StringTok{"X"}\NormalTok{, }\DataTypeTok{y=}\StringTok{""}\NormalTok{, }\DataTypeTok{title =} \StringTok{"Z and Y versus X and estimates"}\NormalTok{)}

\NormalTok{plot3}
\end{Highlighting}
\end{Shaded}

\end{document}
